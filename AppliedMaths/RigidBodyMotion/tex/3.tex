\documentclass{article}
\usepackage{amsmath}
\begin{document}

\title{Kinetic Energy and Angular Momentum of a Rotating Solid Cylinder}
\date{}
\maketitle

Consider a solid cylinder of mass $m$ and radius $a$ rotating with an angular velocity $\omega$ about an axis along its length, i.e., parallel to the axis of symmetry. We aim to find the kinetic energy and angular momentum about this axis.

\section*{Kinetic Energy (KE)}

The kinetic energy of a rotating body is expressed as:
\[ KE = \frac{1}{2} I \omega^2 \]
where $I$ denotes the moment of inertia of the body about the axis of rotation, and $\omega$ represents the angular velocity.

For a solid cylinder rotating about its axis, the moment of inertia $I$ is:
\[ I = \frac{1}{2} m a^2 \]

Therefore, the kinetic energy of the cylinder is calculated as follows:
\[ KE = \frac{1}{2} \left(\frac{1}{2} m a^2\right) \omega^2 = \frac{1}{4} m a^2 \omega^2 \]

\section*{Angular Momentum (L)}

The angular momentum $L$ of a rotating body about the axis of rotation is given by:
\[ L = I \omega \]

For our solid cylinder, utilizing the previously determined moment of inertia $I$, the angular momentum is:
\[ L = \frac{1}{2} m a^2 \omega \]

These formulas provide the kinetic energy and angular momentum of the cylinder in terms of its mass $m$, radius $a$, and angular velocity $\omega$.

\end{document}
