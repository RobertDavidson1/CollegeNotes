\documentclass{article}
\usepackage{amsmath}
\begin{document}

\title{Center of Mass of a Conical Shell}
\date{}
\maketitle

To find the center of mass of a conical shell with radius $a$ and height $h$, we consider the shell to be thin and use the concept of continuous mass distribution. The center of mass ($z_{\text{CM}}$) of the conical shell is calculated by considering an infinitesimally thin circular slice of the cone at a height $z$ from the base, with thickness $dz$ and radius $r(z)$. The relationship between the radius of the slice and its height is linear, given by $r(z) = \frac{a}{h}z$.

The center of mass $z_{\text{CM}}$ is given by:
\[ z_{\text{CM}} = \frac{\int_0^h z \, dm}{\int_0^h dm} \]

The mass of an infinitesimal slice ($dm$) is proportional to its area. Considering the surface area of the infinitesimal slice for a thin shell, the mass distribution is uniform over its surface, hence $dm = \sigma 2\pi r(z) dz$, with $\sigma$ being the surface mass density (mass per unit area) of the shell.

Substituting $r(z) = \frac{a}{h}z$ into the expression for $dm$ gives:
\[ dm = \sigma 2\pi \left(\frac{a}{h}z\right) dz \]

Substituting $dm$ into the integral for $z_{\text{CM}}$, we get:
\[ z_{\text{CM}} = \frac{\int_0^h z \sigma 2\pi \left(\frac{a}{h}z\right) dz}{\int_0^h \sigma 2\pi \left(\frac{a}{h}z\right) dz} \]

Simplifying the integrals:
\[ z_{\text{CM}} = \frac{\int_0^h z^2 dz}{\int_0^h z dz} = \frac{\left[\frac{1}{3}z^3\right]_0^h}{\left[\frac{1}{2}z^2\right]_0^h} = \frac{\frac{1}{3}h^3}{\frac{1}{2}h^2} = \frac{2}{3}h \]

Therefore, the center of mass of the conical shell is located at a distance of $\frac{h}{3}$ from the base along the axis of symmetry, correcting the initial mistake in the explanation.

\end{document}
