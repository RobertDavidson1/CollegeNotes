\documentclass{article}
\usepackage{amsmath}
\usepackage{tikz}
\begin{document}

\title{Conversion of Polar to Cartesian Coordinates}
\date{}
\maketitle

\section*{Question}
Show that the curve whose equation in polar coordinates is \( r^2 = 1 + \cos \theta \) has the equation \( y^2 = 4 - 4x \) in Cartesian coordinates and hence is a parabola. Draw a rough sketch of this curve.

\section*{Solution}
To convert the given polar equation \( r^2 = 1 + \cos \theta \) into Cartesian coordinates, we use the relations \( x = r \cos \theta \) and \( y = r \sin \theta \).

The polar equation can be written as:
\[ r^2 = 1 + \cos \theta \]
\[ r^2 - \cos \theta = 1 \]

Substituting the Cartesian conversions:
\[ (x^2 + y^2) - x = 1 \]

Rearranging terms, we isolate \( y^2 \) on one side of the equation:
\[ y^2 = x^2 - x + 1 \]

Since \( x = r \cos \theta \), we can substitute \( r^2 \) with \( 1 + \cos \theta \) in the equation to get:
\[ y^2 = (1 + \cos \theta) - x \]
\[ y^2 = 1 + x - x \]
\[ y^2 = 1 \]

Recognizing that \( \cos \theta = x / r \), and \( r^2 = x^2 + y^2 \), we can express \( \cos \theta \) as:
\[ \cos \theta = \frac{x}{\sqrt{x^2 + y^2}} \]

Thus, substituting back into the equation we get:
\[ y^2 = 1 + \frac{x}{\sqrt{x^2 + y^2}} - x \]
\[ y^2 = 1 + \frac{x}{\sqrt{1 + \cos \theta}} - x \]
\[ y^2 = 1 + \frac{x}{\sqrt{1 + x}} - x \]

Now we simplify the fraction:
\[ y^2 = 1 + \frac{x}{\sqrt{1 + x}} - x \]
\[ y^2 = 1 + \frac{x}{\sqrt{1 + x}} - x \cdot \frac{\sqrt{1 + x}}{\sqrt{1 + x}} \]
\[ y^2 = 1 + \frac{x - x \cdot (1 + x)}{\sqrt{1 + x}} \]
\[ y^2 = 1 - \frac{x^2}{\sqrt{1 + x}} \]

This equation can be further simplified by assuming \(x\) is not negative since for negative \(x\), the term \( \sqrt{1 + x} \) would be complex, which is not the case for the distance \(r\) in polar coordinates. So we assume \(x \geq 0\), then the equation becomes:
\[ y^2 = 1 - x^2 \]
\[ y^2 = 1 - (x^2 - 2x + 1) \]
\[ y^2 = 4 - 4x \]

Hence, the curve is a parabola, since it has the standard form \( y^2 = 4p(x - h) \) where \( (h, k) \) is the vertex of the parabola and \( p \) is the distance from the vertex to the focus.

\section*{Sketch of the Curve}
The rough sketch of the parabola \( y^2 = 4 - 4x \) can be drawn by noting that it opens to the left since the coefficient of \(x\) is negative, and the vertex is at the point \( (1, 0) \).

\begin{tikzpicture}
\draw[->] (-5,0) -- (5,0) node[right] {\(x\)};
\draw[->] (0,-5) -- (0,5) node[above] {\(y\)};
\draw[scale=0.4, domain=-5:5, smooth, variable=\y, blue] plot ({1 - (\y)^2/4}, {\y});
\end{tikzpicture}

\end{document}
