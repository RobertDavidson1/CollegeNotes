\documentclass{article}
\usepackage{amsmath}
\begin{document}

\title{Calculation of the Martian Year}
\date{}
\maketitle

Given the mass of the Sun as $M = 1.9891 \times 10^{30} \, \text{kg}$, the eccentricity of the elliptical orbit of Mars as $e = 0.093$, and the minimum distance from Mars to the Sun as $1.382 \, \text{AU}$, we aim to calculate the number of Earth years in a Martian year. It is important to note that distances are given in Astronomical Units (AU), where 1 AU is the average distance from the Earth to the Sun, and the units of the universal gravitational constant $G$ are $\text{m}^3\text{kg}^{-1}\text{s}^{-2}$.

\section*{Methodology}

To find the semi-major axis $a$ of Mars' orbit, we use the relationship given by:
\[ a = \frac{r_{\text{min}}}{1 - e} \]
where $r_{\text{min}} = 1.382 \, \text{AU}$ is the perihelion distance (the minimum distance from Mars to the Sun), and $e = 0.093$ is the eccentricity of Mars' orbit.

Next, applying Kepler's Third Law, which relates the orbital period of a planet to the semi-major axis of its orbit:
\[ T^2 = \frac{4\pi^2}{G(M+m)}a^3 \]
Assuming the mass of Mars $m$ is negligible compared to the mass of the Sun $M$, and converting $a$ from AU to meters (knowing $1 \, \text{AU} = 1.496 \times 10^{11} \, \text{m}$), we can calculate the orbital period $T$ of Mars.

\section*{Calculation}

Substituting the given values into the equation for $a$:
\[ a = \frac{1.382}{1 - 0.093} = \frac{1.382}{0.907} \, \text{AU} \]

Then, converting the semi-major axis to meters and substituting into Kepler's Third Law:
\[ T^2 = \frac{4\pi^2}{G(M+m)}\left(a \times 1.496 \times 10^{11}\right)^3 \]

Given $G = 6.674 \times 10^{-11} \, \text{m}^3\text{kg}^{-1}\text{s}^{-2}$ and the mass of the Sun, we can find $T$ in seconds and convert this to Earth years by dividing by $3.156 \times 10^7 \, \text{s/year}$.

\section*{Conclusion}

This process allows us to calculate the number of Earth years in a Martian year, using the given orbital parameters of Mars and applying Kepler's laws of planetary motion.

\end{document}
