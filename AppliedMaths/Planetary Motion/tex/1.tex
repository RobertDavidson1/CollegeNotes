\documentclass{article}
\usepackage{amsmath}
\begin{document}

\title{Calculation of Mercury's Maximum Distance from the Sun}
\date{}
\maketitle

\section*{Question}
The eccentricity of the orbit of the planet Mercury is 0.205 and its average distance from the Sun is 0.387 AU (Astronomical Units). What is its maximum distance from the Sun? (Note: One AU is approximately 149.6 million kilometres. It is close to the average distance from the Earth to the Sun. The AU is often used as a unit of distance within the solar system.)

\section*{Solution}
To find the maximum distance of Mercury from the Sun, we use the properties of its elliptical orbit. The orbit's eccentricity, denoted by \(e\), is given as 0.205. The average distance from the Sun, which corresponds to the semi-major axis (\(a\)), is 0.387 AU.

The formula to calculate the maximum distance (\(d_{\text{max}}\)) from the planet to the Sun, or the aphelion distance, in an elliptical orbit is:
\[ d_{\text{max}} = a(1 + e) \]

Where:
\begin{itemize}
    \item \(a\) is the semi-major axis of the orbit (average distance from the Sun), and
    \item \(e\) is the eccentricity of the orbit.
\end{itemize}

Given that Mercury's semi-major axis \(a = 0.387 \, \text{AU}\) and its eccentricity \(e = 0.205\), we substitute these values into the formula:
\[ d_{\text{max}} = 0.387 \times (1 + 0.205) \]

Calculating the expression gives us Mercury's maximum distance from the Sun in Astronomical Units.

\end{document}
