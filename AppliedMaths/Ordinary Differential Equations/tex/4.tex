\documentclass{article}
\usepackage{amsmath}
\begin{document}

\title{Newton's Law of Cooling Problem}
\date{}
\maketitle

\section*{Problem Statement}
Coffee at a temperature of 85°C is cooling in a room at constant temperature of 20°C. It takes 1 minute for the temperature to reach 75°C. Based on Newton's law of cooling, find:
\begin{enumerate}
    \item[(a)] the coffee’s temperature after 3 minutes;
    \item[(b)] the time it takes for the coffee to reach a temperature of 45°.
\end{enumerate}

\section*{Solution}
Newton's Law of Cooling is given by the differential equation:
\[ \frac{dT}{dt} = -k(T - T_{\text{ambient}}) \]

\subsection*{Part (a)}
We use the initial condition \( T(0) = 85 \)°C to find the constant \( k \). When \( T = 75 \)°C, we solve for \( k \):
\[ \frac{dT}{dt} = -k(75 - 20) \]
Since the temperature decreases by 10 degrees in 1 minute, we have:
\[ -k \times 55 = \frac{-10}{1} \]
\[ k = \frac{10}{55} \]

Using this \( k \), the solution to the differential equation is:
\[ T(t) = T_{\text{ambient}} + (T(0) - T_{\text{ambient}})e^{-kt} \]
\[ T(3) = 20 + (85 - 20)e^{-\frac{10}{55} \times 3} \]

\subsection*{Part (b)}
To find the time it takes for the coffee to reach 45°C, we set \( T(t) = 45 \)°C and solve for \( t \):
\[ 45 = 20 + (85 - 20)e^{-\frac{10}{55}t} \]
\[ 25 = 65e^{-\frac{10}{55}t} \]
\[ e^{-\frac{10}{55}t} = \frac{25}{65} \]
\[ -\frac{10}{55}t = \ln\left(\frac{25}{65}\right) \]
\[ t = -\frac{55}{10}\ln\left(\frac{25}{65}\right) \]

\section*{Conclusion}
We have calculated the coffee's temperature after 3 minutes and the time it takes to reach 45°C using Newton's Law of Cooling.

\end{document}
