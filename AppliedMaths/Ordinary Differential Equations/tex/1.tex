\documentclass{article}
\usepackage{amsmath}
\usepackage{amssymb}
\begin{document}

\title{Bacterial Population Growth}
\date{}
\maketitle

\section*{Problem Statement}
Calculate how long it will take for a population of unicellular bacteria, starting from 1 bacterium and dividing every 20 minutes, to cover the Earth's surface with a layer of one meter deep according to the Malthusian model.

\section*{Solution}
The growth of the bacterial population is described by the Malthusian model of exponential growth:
\[ P(t) = P_0 e^{rt} \]
where \( P(t) \) is the population at time \( t \), \( P_0 \) is the initial population, and \( r \) is the growth rate.

Given the doubling time \( T_d = 20 \) minutes, the growth rate \( r \) is:
\[ r = \frac{\ln(2)}{T_d} \]

To find the time \( t \) when the population reaches a certain size, we rearrange the growth equation:
\[ t = \frac{\ln(P(t) / P_0)}{r} \]

The Earth's surface area is approximately \( 510.1 \times 10^6 \) km\(^2\), and we wish to cover it with a layer 1 meter deep. Converting the area to m\(^2\):
\[ \text{Surface area} = 510.1 \times 10^6 \times 10^6 \text{ m}^2 \]

Assuming each bacterium occupies a volume of \( 1 \mu m^3 \), or \( 1 \times 10^{-18} \text{ m}^3 \), the number of bacteria needed to cover the Earth with a 1-meter layer is:
\[ N = \frac{\text{Surface area}}{\text{Bacterium volume}} \]
\[ N = \frac{510.1 \times 10^{12}}{1 \times 10^{-18}} \]

The initial population \( P_0 = 1 \). We now solve for \( t \):
\[ t = \frac{\ln(N / P_0)}{r} \]
\[ t = \frac{\ln(510.1 \times 10^{30})}{\frac{\ln(2)}{20}} \]

This will give us \( t \) in minutes. To convert \( t \) to more practical units, such as days or years, we will use appropriate conversion factors.

\section*{Conclusion}
By calculating \( t \), we will determine the time required for the bacterial population to cover the Earth's surface with a layer one meter deep, assuming the Malthusian growth model with no limitations on resources or space.

\end{document}
