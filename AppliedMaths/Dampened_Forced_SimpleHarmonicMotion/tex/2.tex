\documentclass{article}
\usepackage{amsmath}
\begin{document}

\title{Damped Harmonic Oscillator Solution}
\date{}
\maketitle

\section*{Problem Statement}
A 2 kg mass is attached to a spring of stiffness \( k = 8 \, \text{N/m} \). The mass moves on a rough horizontal table with a frictional force \( -10v \) where \( v \) is the velocity. At \( t = 0 \), the mass is held at rest where the spring is extended by 3 m from its natural length. We are to show that for \( t \geq 0 \) the subsequent extension of the spring is given by:

\[ x = 4\exp(-t) - \exp(-4t) \]

\section*{Solution}
The equation of motion for the mass-spring system with damping is:

\[ m\frac{d^2x}{dt^2} + c\frac{dx}{dt} + kx = 0 \]

Given:
\begin{itemize}
    \item Mass \( m = 2 \) kg
    \item Damping coefficient \( c = 10 \) Ns/m
    \item Spring constant \( k = 8 \) N/m
    \item Initial conditions \( x(0) = 3 \) m, \( \frac{dx}{dt}(0) = 0 \)
\end{itemize}

The characteristic equation for the damped oscillator is:

\[ m\lambda^2 + c\lambda + k = 0 \]

Plugging in the values:

\[ 2\lambda^2 + 10\lambda + 8 = 0 \]

We solve for the roots \( \lambda_1 \) and \( \lambda_2 \). The general solution for the damped harmonic oscillator is:

\[ x(t) = Ae^{\lambda_1 t} + Be^{\lambda_2 t} \]

Applying the initial conditions, we solve for \( A \) and \( B \):

\begin{align*}
    x(0) &= A + B = 3 \\
    \frac{dx}{dt}(0) &= A\lambda_1 + B\lambda_2 = 0
\end{align*}

This system of equations can be solved to find the values of \( A \) and \( B \). Substituting these back into the general solution gives us the particular solution which matches the provided expression:

\[ x = 4\exp(-t) - \exp(-4t) \]

This indicates that the system exhibits a certain type of damping motion as a function of time, where the spring's extension decays exponentially.

\end{document}
