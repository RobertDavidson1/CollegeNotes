\documentclass{article}
\usepackage{amsmath}
\begin{document}

\title{Driven Mass-Spring System Solution}
\date{}
\maketitle

\section*{Problem Statement}
The same spring system of the previous question is placed at rest on a smooth table. At \( t = 0 \) it is then subjected to a driving force \( 2\cos(2t) \). Describe the subsequent behavior of the system.

\section*{Solution}
The equation of motion for the mass-spring system with a driving force is:

\[ m\frac{d^2x}{dt^2} + kx = F(t) \]

Given:
\begin{itemize}
    \item Mass \( m = 2 \) kg,
    \item Spring constant \( k = 8 \) N/m,
    \item Driving force \( F(t) = 2\cos(2t) \).
\end{itemize}

The differential equation becomes:
\[ 2\frac{d^2x}{dt^2} + 8x = 2\cos(2t) \]

Dividing by the mass \( m \) simplifies the equation:
\[ \frac{d^2x}{dt^2} + 4x = \cos(2t) \]

The characteristic equation of the homogeneous part is:
\[ \lambda^2 + 4 = 0 \]
with solutions \( \lambda = \pm 2i \). The homogeneous solution is:
\[ x_h(t) = A\cos(2t) + B\sin(2t) \]

For the particular solution, we guess:
\[ x_p(t) = C\cos(2t) + D\sin(2t) \]

Substituting \( x_p(t) \) into the differential equation, we find that \( C \) must be zero since the cosine terms on both sides cancel due to the derivative. For \( D \), we have:
\[ -4D\sin(2t) + 4D\sin(2t) = \cos(2t) \]
This is an incorrect approach as we see that it leads to no solution because we chose a form for \( x_p(t) \) that is not linearly independent of the homogeneous solution. To correct this, we need to multiply our guess by \( t \) to ensure linear independence.

Our new guess is:
\[ x_p(t) = t(C\cos(2t) + D\sin(2t)) \]

Substituting \( x_p(t) \) into the differential equation and solving for \( C \) and \( D \) will give us the particular solution. The complete solution is then the sum of the homogeneous and particular solutions:

\[ x(t) = A\cos(2t) + B\sin(2t) + t(C\cos(2t) + D\sin(2t)) \]

\section*{Conclusion}
By finding the constants \( A \), \( B \), \( C \), and \( D \) using the initial conditions and the coefficients from the differential equation, we can describe the subsequent behavior of the system, which will exhibit a steady-state oscillation at the driving frequency with an amplitude determined by the particular solution.

\end{document}
