\documentclass{article}
\usepackage{amsmath}
\begin{document}

\title{Extension of a Spring in Circular Motion}
\date{}
\maketitle

\section*{Problem Statement}
A particle of mass 1 kg is attached to one end of an elastic spring of natural length 1 m and modulus of elasticity 50 N. The other end is fastened to a point on a smooth horizontal table. If the spring and particle describe circles on the table at 60 revolutions per minute, find the extension of the spring.

\section*{Solution}
The centripetal force required for circular motion is provided by the force exerted by the spring. The centripetal force \( F \) is given by \( F = \frac{mv^2}{r} \), and the force by the spring follows Hooke's Law \( F = kx \), where \( k \) is the spring constant and \( x \) is the extension of the spring.

Given:
\begin{itemize}
    \item Mass of the particle \( m = 1 \) kg,
    \item Spring constant \( k = 50 \) N/m,
    \item Revolutions per minute \( n = 60 \).
\end{itemize}

First, we convert the revolutions per minute to angular velocity in radians per second:
\[ \omega = 2\pi \times \frac{n}{60} \, \text{rad/s} = 2\pi \, \text{rad/s} \]

The radius of the motion \( r \) is equal to the extension \( x \) since the natural length does not provide any force. The centripetal force is therefore:
\[ m(\omega x)^2 = kx \]
Solving for \( x \) gives us:
\[ x = \frac{m\omega^2}{k} \]

Substituting the values we have:
\[ x = \frac{(1 \, \text{kg})(2\pi \, \text{rad/s})^2}{50 \, \text{N/m}} \]
\[ x = \frac{(1)(4\pi^2)}{50} \, \text{m} \]
\[ x = \frac{4\pi^2}{50} \, \text{m} \]

\section*{Conclusion}
The extension \( x \) of the spring is calculated to be \( \frac{4\pi^2}{50} \) meters when the particle describes circles at a rate of 60 revolutions per minute.

\end{document}
