\documentclass{article}
\usepackage{amsmath}
\usepackage{amssymb}
\begin{document}

\title{Banked Curved Road Problem}
\date{}
\maketitle

\section*{Problem Statement}
A car travels on a rough banked curved road of radius 200 m with velocity \( v \) where the road is inclined at an angle of \( 15^\circ \) to the horizontal. The coefficient of friction between the road and the car is \( \frac{1}{4} \). Find the value of \( v \) if:
\begin{enumerate}
    \item[(a)] the car is just about to slip up the road;
    \item[(b)] the car is just about to slip down the road.
\end{enumerate}

\section*{Solution}
To solve for \( v \), we balance the forces acting on the car. The gravitational force \( mg \) has a component \( mg\sin(\theta) \) acting down the incline, and a component \( mg\cos(\theta) \) acting normal to the incline. The normal force \( N \) is equal to \( mg\cos(\theta) \), and the maximum frictional force \( f \) is \( \mu N \), where \( \mu \) is the coefficient of friction.

\subsection*{Part (a)}
For the car just about to slip up the road, the frictional force \( f \) acts down the incline, opposing the centripetal force. The centripetal force needed to keep the car in circular motion is given by:
\[ f + N\sin(\theta) = \frac{mv^2}{r} \]
The frictional force is \( f = \mu N = \mu mg\cos(\theta) \). Substituting \( N \) and \( f \) we get:
\[ \mu mg\cos(\theta) + mg\cos(\theta)\sin(\theta) = \frac{mv^2}{r} \]
\[ \left( \mu + \sin(\theta) \right) g\cos(\theta) = \frac{v^2}{r} \]
Solving for \( v \) gives us:
\[ v = \sqrt{\left( \mu + \sin(\theta) \right) g\cos(\theta) \cdot r} \]

\subsection*{Part (b)}
For the car just about to slip down the road, the frictional force \( f \) acts up the incline, contributing to the centripetal force. The equation now becomes:
\[ N\sin(\theta) - f = \frac{mv^2}{r} \]
\[ mg\cos(\theta)\sin(\theta) - \mu mg\cos(\theta) = \frac{mv^2}{r} \]
\[ \left( \sin(\theta) - \mu \right) g\cos(\theta) = \frac{v^2}{r} \]
Solving for \( v \) gives us:
\[ v = \sqrt{\left( \sin(\theta) - \mu \right) g\cos(\theta) \cdot r} \]

\section*{Values}
Plugging in \( \theta = 15^\circ \), \( \mu = \frac{1}{4} \), \( g = 9.8 \, \text{m/s}^2 \), and \( r = 200 \, \text{m} \), we can compute the specific values for \( v \) in both scenarios.

\end{document}
