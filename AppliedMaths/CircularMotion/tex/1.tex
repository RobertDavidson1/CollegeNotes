\documentclass{article}
\usepackage{amsmath}
\begin{document}

\title{Centripetal Force and Frictional Force Calculations}
\date{}
\maketitle

\section*{Problem Statement}
A car travels at constant speed around a horizontal circular corner of radius 5 m.
\begin{enumerate}
    \item[(a)] Given that the car just starts to skid if its speed is 12 km/h, find the frictional force acting on the car.
    \item[(b)] Assuming the same frictional force is acting, calculate the car's smallest possible turning radius if the speed is 30 km/h.
    \item[(c)] Calculate the turning radius for the car traveling at 12 km/h in wet conditions where the frictional force is halved.
\end{enumerate}

\section*{Solution}
\subsection*{Part (a)}
The frictional force provides the centripetal force required to keep the car moving in a circle. The centripetal force is given by:
\[ F = \frac{mv^2}{r} \]
To find the frictional force \( F \), we first convert the speed from km/h to m/s:
\[ v = 12 \, \text{km/h} = \frac{12 \times 1000}{3600} \, \text{m/s} \]
\[ v = \frac{10}{3} \, \text{m/s} \]
The radius \( r \) of the circular path is 5 m. Assuming the mass of the car is \( m \) (which is not provided and not necessary for the calculation as we will see), the frictional force \( F \) at the point of skidding is:
\[ F = \frac{mv^2}{r} \]
\[ F = \frac{m\left(\frac{10}{3}\right)^2}{5} \, \text{N} \]
\[ F = \frac{100m}{9 \times 5} \, \text{N} \]
\[ F = \frac{20m}{9} \, \text{N} \]

\subsection*{Part (b)}
The frictional force calculated in part (a) remains the same, and we want to find the smallest possible turning radius \( r \) for a speed of 30 km/h. First, we convert the speed:
\[ v = 30 \, \text{km/h} = \frac{30 \times 1000}{3600} \, \text{m/s} \]
\[ v = \frac{25}{3} \, \text{m/s} \]
Using the same frictional force \( F \) from part (a), we find the new radius \( r \):
\[ F = \frac{mv^2}{r} \]
\[ \frac{20m}{9} = \frac{m\left(\frac{25}{3}\right)^2}{r} \]
\[ \frac{20}{9} = \frac{625}{9r} \]
\[ r = \frac{625}{20} \, \text{m} \]
\[ r = 31.25 \, \text{m} \]

\subsection*{Part (c)}
In wet conditions, the frictional force is halved, so we have:
\[ F = \frac{20m}{9 \times 2} \, \text{N} \]
Using the original speed of 12 km/h (\( \frac{10}{3} \, \text{m/s} \)), we calculate the new turning radius \( r \):
\[ \frac{20m}{18} = \frac{m\left(\frac{10}{3}\right)^2}{r} \]
\[ \frac{20}{18} = \frac{100}{9r} \]
\[ r = \frac{100}{10} \, \text{m} \]
\[ r = 10 \, \text{m} \]

\section*{Conclusion}
By analyzing the frictional force required for centripetal motion, we have calculated the frictional force acting on a car turning at 12 km/h, the smallest turning radius possible at 30 km/h assuming the same frictional force, and the turning radius at 12 km/h when the frictional force is halved due to wet conditions.

\end{document}
